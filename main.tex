\documentclass{article}

\usepackage{sectsty}  % \sectionfont
\usepackage{titlecaps}  % \titlecap
\sectionfont{\titlecap}  % capitalize section title

% use geometry to set paper size in US letter
\usepackage[letterpaper]{geometry}

% support multiple color keyword
\usepackage{xcolor}

% text highlighting, underlining etc..
\usepackage{soul}

% Use minted to show code, outputdif should be the same as option 
% --output-directories passed to latex
\usepackage[outputdir=build/]{minted}
\usemintedstyle{github}
\definecolor{bg}{HTML}{F7F3DD}
% remove leading spaces for all languages
\setminted{bgcolor=bg, autogobble}
% can also write in specific language
\setminted[latex]{autogobble}

% Show fancy code font, requires xelatex
\usepackage{fontspec}
\setmonofont{JetBrainsMono}[
  Path=./fonts/JetBrainsMono/,
  Scale=MatchLowercase,
  Extension = .ttf,
  UprightFont=*-Regular,
  BoldFont=*-Bold,
  ItalicFont=*-Italic,
  BoldItalicFont=*-BoldItalic
]


\usepackage{parskip} % paragraph with have no indent but wider vertical gap
\usepackage{graphicx} % figures
\usepackage{hyperref}
\usepackage{csquotes}
\usepackage{enumitem} % enumerate itemsep
\hypersetup{
  colorlinks=true,
  linkcolor=black,
  urlcolor=cyan,
  citecolor=blue,
}

\title{\LaTeX\ Writing Guide}
\author{Kaidong Hu}

\begin{document}
\maketitle


\section{Start a latex project}
\subsection{Scaffold}
Following code is the scaffold of a latex project.
\begin{minted}{latex}
  \documentclass{article}
  \title{\LaTeX\ writing guide}
  \author{Kaidong Hu}

  \begin{document}
  \maketitle

  \end{document}
\end{minted}

\subsection{Vim modeline}
Put following lines at the end of every tex file will help for a fluent vim 
editing.
\begin{minted}{latex}
% vim:set fo+=wat fo-=l :
% vim:set fdm=marker fdl=1 fmr=<<<,>>>:
\end{minted}

\section{Preferred packages}
\inputminted{latex}{header-preferred.tex}

\section{Including images}
To include an image, following header package is required:
\begin{minted}{latex}
\usepackage{graphicx}
\end{minted}

and following command is used to include an image
\begin{minted}{latex}
\includegraphics[width=\textwidth]{abc.png}
\end{minted}

\section{Code listings}

Install \mintinline{bash}{pygments} so latex could use it to highlight codes

\begin{minted}{bash}
  pipx install pygments
\end{minted}

\inputminted{latex}{header-codelisting.tex}

In order to use github style, install it using

\begin{minted}{bash}
  pipx runpip pygments install pygments-style-github
\end{minted}


\section{Code listing alternative}

Here this \href{https://tex.stackexchange.com/questions/452150/
advanced-customization-of-minted-code}{Stackoverflow solution} shows a pretty 
demo of highlighted code. Could be researched later.
\includegraphics[width=\textwidth]{images/tSM9l.png}

\section{Paper citations}

\subsection{Citing with bibtex}
This is the preferred way when creating a journal or conference paper.

\subsection{Citing with biblatex and biber}
This is the preferred way to create a personal record. Following code need to 
put in the header of a latex document
\begin{minted}{latex}
  \usepackage[
    citestyle=alphabetic,
    bibstyle=alphabetic,
    sorting=ynt
  ]{biblatex}

  \addbibresource{main.bib}
\end{minted}
and an extra line will be inserted inside the document body
\begin{minted}{latex}
\printbibliography
\end{minted}

Following commands are provided to cite a reference:
\begin{enumerate}[itemsep=0mm]
  \item \mintinline{latex}/\cite{*}/
  \item \mintinline{latex}/\autocite{*}/
  \item \mintinline{latex}/\textcite{*}/
  \item \mintinline{latex}/\citeauthor{*}/
  \item \mintinline{latex}/\citetitle{*}/
  \item Suffixing a \* after \mintinline{latex}/\cite*{*}/ will make it a 
    plain-text citation without create links and references.
\end{enumerate}

\subsection{Biblographies per chapter}
By setting \mintinline{latex}{[refsection=chapter]} command in biblatex,
Following declarasion can be set in document permeable part to make bibliography consists with other part of chapter.  
Using
\begin{minted}{latex}
\DeclarePrintbibliographyDefaults{
  heading=subbibintoc,
  title=See also:
}
\end{minted}

One can add citation in each chapter and put \mintinline{latex}{\printbibliography} at the end of each chapter.

\section{hyperref and link colors}
\begin{minted}{latex}
\usepackage{hyperref}
\hypersetup{
  colorlinks=true,
  linkcolor=black,
  urlcolor=cyan,
  citecolor=blue,
}
\end{minted}

\section{paragraph quotation}
"csquotes" package showing words in following format:
\begin{displayquote}
  \begin{centering}
  \enquote{\it something quoted}
  \end{centering}
\end{displayquote}
\begin{minted}{latex}
\usepackage{csquotes}

% if bible included
\usepackage[bable=false]{csquotes}

\begin{displayquote}
  \begin{centering}
  \enquote{\it something quoted}
  \end{centering}
\end{displayquote}
\end{minted}

\section{Put float here!}

The first choice is a \mintinline{latex}{\clearpage}. command. It typesets 
every floats till command issued and put a \mintinline{latex}{\newpage} 
afterward.

The second choice is \mintinline{latex}{\FloatBarrier} provided by 
\mintinline{latex}{placeins} package. It flushes all floats without creating a 
new page.

\end{document}


% vim:set fo+=wat fo-=l :
% vim:set fdm=marker fdl=1 fmr=<<<,>>>:
